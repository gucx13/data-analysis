

\documentclass{article}
\usepackage{CJK}
\usepackage{amsmath,amssymb}
\usepackage{fancyhdr}  


\begin{document}
\begin{CJK*}{GBK}{song}

\pagestyle{fancy}  
\fancyhead{} % clear all fields  
\fancyhead[R]{Particle Physics}  
\fancyhead[L]{Chenxi Gu\\ 2017311017} 
\renewcommand{\headrulewidth}{0.4pt}  
\renewcommand{\footrulewidth}{0.4pt} 



\title {chapter 1}
\author{Chenxi Gu\\2017311017}

\date{\today}

\maketitle
\section{Problem 1}
The energy of a Boeing 747:
\begin{center}
$E=\frac{p^2}{2m}=1.11*10^{10}J$
\end{center}

The energy of mosquito–antimosquito annihilation:
\begin{center}
$E=2mc^2=3.6*10^{11}J$
\end{center}

\section{Problem 2}
\begin{equation}
\begin{aligned}
&s=(3E)^2-0^2=9(m^2+p^2)=88.9GeV^2\\
&m=\sqrt{s}=9.43GeV
\end{aligned}
\end{equation}


\section{Problem 3}
We know the decay width $\Gamma=\frac{\hbar}{\tau}$.
\begin{center}
$
\Gamma_{\pi}=2.538*10^{-8}eV,\quad \Gamma_{K}=5.5*10^{-8}eV,\quad \Gamma_{\Lambda}=2.538*10^{-6}eV
$
\end{center}

\section{Problem 4}
Same reason with Problem 3
\begin{center}
$
\tau_{\rho}=4.429*10^{-24}s,\quad \tau_{\omega}=7.765*10^{-23}s,\quad \tau_{\phi}=1.535*10^{-22}s
$
$
\tau_{K^{*}}=1.294*10^{-23}s,\quad \tau_{J/\psi}=7.097*10^{-21}s,\quad\tau_{\Delta}=5.593*10^{-24}s
$
\end{center}

\section{Problem 5}
\begin{equation}
E=\sqrt{p^2+m^2}
\end{equation}
The momentum of electron beam are:$p=20GeV/c$
The angular change of the electron beam is $6^{\circ}$. So momentum transfer is $\Delta p= 2.094GeV/c$.
According to uncertainty principle:
\begin{equation}
\Delta x\Delta p\sim \hbar 
\end{equation}
\begin{center}
$
\Delta x=9.453*10^{-17}
$
\end{center}

\section{problem 6}
(a)  
Because of energy conservation law:
\begin{equation}
\sqrt{p^2c^2+m^2_pc^4}+m_pc^2=\sqrt{(2m_p+m)^2c^4+p^2c^2}
\end{equation}
So we can solve the threshold energy and momentum:
\begin{equation}
\begin{aligned}
&E_p=\frac{(2m_p^2+4m_pm+m^2)c^2}{2m_p}\\
&p=\frac{\sqrt{E_p^2-m_p^2c^4}}{c}
\end{aligned}
\end{equation}
(b)
\begin{equation}
\begin{aligned}
&E^*_p=\frac{(2m_p+m)c^2}{2}\\
&p=\frac{\sqrt{E_p^2-m_p^2c^4}}{c}
\end{aligned}
\end{equation}
(c) We know the pion is $\pi^0$, and it's mass is 135$MeV/c^2$.
\begin{equation}
\begin{aligned}
E_p=1217.7MeV,\quad E^*_p=1005.5MeV
\end{aligned}
\end{equation}
The kinetic energy in case (a):
\begin{equation}
K=E_p-m_pc^2=279.7MeV
\end{equation}

\section{Problem 7}
(a) Because of energy conservation law:
\begin{equation}
pc+m_pc^2=\sqrt{p^2c^2+(m_p+m_{\pi^0})^2c^4}
\end{equation}
We solve the threshold energy:
\begin{equation}
E_{\gamma}=144.7MeV
\end{equation}

(b)
The collision between photon and proton must head to head.
\begin{equation}
p_{\gamma}c+\sqrt{p^2c^2+m_p^2c^4}=\sqrt{(pc-p_{\gamma}c)^2+(m_p+m_{\pi^0})^2c^4}
\end{equation}
using Mathematica we can calculate $E_p=6.78*10^{13}MeV$.\\

(c)
The attenuation length$L=\frac{1}{\sigma\rho}=1.67*10^{23}m$, So we compare it  with light-year:$L=1.76*10^{7}l.y.$. 

\section{Problem 8}
(a)
\begin{center}
$
E_{\gamma}=2.61*10^8MeV
$
\end{center}
(b)
The EBL photons energy is $E=1.24eV$
\begin{center}
$
E_{\gamma}=2.61*10^5MeV
$
\end{center}

\section{Problem 9}
The minimum energy is the head to head collision situation.
\begin{equation}
E_p=2m_pc^2=1.876GeV
\end{equation}

\section{Problem 10}
Consider Lorenz invariant s:
\begin{equation}
\begin{aligned}
&s=(E_{\gamma_1}+E_{\gamma_2})^2-(p_{\gamma_1}+p_{\gamma_2})^2\\
&E_p=1.044*10^{11}MeV
\end{aligned}
\end{equation}

\section{Problem 11}
\begin{equation}
\sqrt{m_1^2+p^2}+\sqrt{m_2^2+p^2}=M
\end{equation}
We can infer:
\begin{equation} 
\begin{aligned}
&E_{m_1}=\frac{M^2+m_1^2-m_2^2}{2M}\\
&E_{m_2}=\frac{M^2+m_2^2-m_1^2}{2M}\\
&p=\sqrt{E_{m_1}^2-m_1^2}
\end{aligned}
\end{equation}

\section{Problem 12}
Decay $\Lambda\rightarrow p\pi^-$:
\begin{equation} 
\begin{aligned}
&M=1115.6MeV\\
&m_1=938.3MeV\\
&m_2=139.5MeV
\end{aligned}
\end{equation}

\begin{equation} 
\begin{aligned}
&E_{m_1}=943.6MeV\\
&E_{m_2}=171.9MeV\\
&p=100.4MeV/c
\end{aligned}
\end{equation}
Decay $\Xi^-\rightarrow\Lambda\pi^-$
\begin{equation} 
\begin{aligned}
&M=1321.7MeV\\
&m_1=1115.6MeV\\
&m_2=139.5MeV
\end{aligned}
\end{equation}

\begin{equation} 
\begin{aligned}
&E_{m_1}=1124.3MeV\\
&E_{m_2}=197.4MeV\\
&p=139.6MeV/c
\end{aligned}
\end{equation}

\section{Problem 13}
We could use the result in Problem 11.
\begin{equation} 
\begin{aligned}
&E_{m_1}=\frac{M^2+m_1^2}{2M}\\
&E_{m_2}=\frac{M^2-m_1^2}{2M}\\
&p=\frac{M^2-m_1^2}{2Mc}
\end{aligned}
\end{equation}

\section{Problem 14}
We consider critical situation:$\mu$ at rest, neutrino carry all momentum.Because mass of Neutrino is very small, we treat it like photon.
\begin{equation} 
\begin{aligned}
&\sqrt{p_{\pi}^2+m_{\pi}^2}=m_{\mu}+p_{\pi}\\
&p_{\pi}=39.3MeV/c
\end{aligned}
\end{equation}
\section{problem 15}
(a) We can use result from Problem 12:
\begin{equation} 
\begin{aligned}
&E_{\Lambda}=1115.6MeV\\
&E_{\pi}=171.9MeV\\
&p_{\Lambda}=0MeV/c\\
&p_{\pi}=100.4MeV/c\\
\end{aligned}
\end{equation}


(b)
\begin{equation}
\begin{aligned}
&E_{\Lambda}=\sqrt{p_{\Lambda}^2+m_{\Lambda}^2}=2.29GeV\\
&\gamma_{\Lambda}=\frac{E_{\Lambda}}{m_{\Lambda}}=2.05\\
&\beta_{\Lambda}=\frac{p_{\Lambda}}{E_{\Lambda}}=0.873\\
\end{aligned}
\end{equation}
(c)
We use the Lorenz transformation:
\begin{equation}
\begin{aligned}
&E^{'}=\gamma \beta p_x+\gamma E\\
&E_{lab,\pi}=196.8MeV,
&p_{lab,\pi}=138.8MeV/c
\end{aligned}
\end{equation}
similarly, we can calculate the proton momentum.
\begin{equation}
\begin{aligned}
&E_{lab,p}=2089.9MeV\\
&p_{lab,p}=1867.7MeV/c
\end{aligned}
\end{equation}
we know the $p_y $ is a Lorenz invariant.
\begin{equation}
\begin{aligned}
&p_y=p_y^{'}sin(\theta)=50.2MeV/c\\
&\theta=arcsin(\frac{p_y}{p})=1.54^{\circ}
\end{aligned}
\end{equation}

\section{Problem 16}
The angular between two final directions is $90^{\circ}$

\section{Problem 17}
\begin{equation}
\begin{aligned}
&E_p=\sqrt{p^2+m^2}=3.14GeV\\
&s=(E_1+E_2)^2-(p_1+p_2)^2=7.66GeV^2
\end{aligned}
\end{equation}
Because s is a lorentz invariant:
\begin{equation}
\begin{aligned}
&E_{p,CM}=1.383GeV\\
&\gamma_{CM}=1.47, \beta_{CM}=0.735
\end{aligned}
\end{equation}

relativity momentum transformation:
\begin{equation}
\begin{aligned}
&p^{'}_y=p_y=0.176GeV/c\\
&p_{1x}^{'}=p_{1x}\gamma+p_{10}\gamma\beta=2.96GeV/c\\
&p_{2x}^{'}=p_{2x}\gamma-p_{20}\gamma\beta=0.0231GeV/c
\end{aligned}
\end{equation}
\begin{equation}
\theta=arctan(\frac{p^{'}_y}{p^{'}_{1x}})+arctan(\frac{p^{'}_y}{p^{'}_{2x}})=85.9^{\circ}
\end{equation}

\section{Problem 18}
We know $m_{D^0}=1864.8MeV.$
\begin{equation}
\beta=\sqrt{1-(\frac{M}{E})^2}=0.9981
\end{equation}
And we know the proper time relation:
\begin{equation}
\begin{aligned}
&d=\frac{ct\beta}{\sqrt{1-\beta^2}}\\
&t=6.21*10^{-13}
\end{aligned}
\end{equation}
we could use the result from Problem 11:

\begin{equation}
\begin{aligned}
E_{\pi^+}=871.2MeV
p_{\pi^+}=859.9MeV/c
\end{aligned}
\end{equation}

\section{Problem 19}
Consider relativity effect:
\begin{equation}
\begin{aligned}
&\frac{ct\beta}{\sqrt{1-\beta^2}}=l\\
&\beta=0.99915
\end{aligned}
\end{equation}
use relativity mass-energy relation:
\begin{equation}
\begin{aligned}
&p_{\pi^-}=3385MeV/c\\
&E_{\pi^-}=3388MeV
\end{aligned}
\end{equation}


\section{Problem 20}
We know $m_p=938.3MeV, m_n=939.5MeV, m_{\pi^-}=139.5MeV, m_{\pi^0}=134.9MeV. $\\
We could use result from Problem 11.
\begin{equation}
\begin{aligned}
&E_{n}=\frac{M^2+m_n^2-m_{\pi^0}^2}{2M}=939.9MeV\\
&E_{\pi^0}=\frac{M^2-m_n^2+m_{\pi^0}^2}{2M}=137.8MeV\\
&K_{n}=E_{n}-m_n=0.43MeV
\end{aligned}
\end{equation}
according to $E_{\pi^0}=\frac{m_{\pi^0}}{\sqrt{1-\beta^2}}$,we can get $\beta_{\pi^0}=0.2,l_{\pi^0}=5nm$

\section{Problem 27}
The relation is:
\begin{equation}
p=BqR
\end{equation}


\section{Problem 28}
\begin{equation}
10^3\sigma l \rho N_A=\frac{N_0-N_H}{N_0}
\end{equation}
So $\sigma=2.22*10^{-30}m^2$


\section{Problem 29}
\begin{equation}
\begin{aligned}
&\beta=\sqrt{\frac{p^2}{p^2+m^2}}\\
&\beta=0.787
\end{aligned}
\end{equation}
Cherenkov threshold is that $n>\frac{1}{\beta}=1.27$.\\
If the index is 1.5, the Cherenkov angle is $\theta=arccos(\frac{1}{n\beta})=32.1^{\circ}$

\section{Problem 30}
According to the relativity mass-momentum relation:
\begin{equation}
\beta=\sqrt{\frac{p^2}{p^2+m^2}}
\end{equation}

So we can get $\Delta t$:
\begin{equation}
\Delta t=2L(\sqrt{1+\frac{m_1^2}{p^2}}-\sqrt{1+\frac{m_2^2}{p^2}})
\end{equation}

\begin{equation}
L_{min}=25.4m
\end{equation}



\section{Problem 32}
We know the red light wavelength is 700nm, green light wavelength is 500nm.\\
use the Doppler effect formula:
\begin{equation}
\begin{aligned}
&\frac{\lambda_{r}}{\lambda_{g}}=\sqrt{\frac{1+\beta}{1-\beta}}\\
&\beta=0.324
\end{aligned}
\end{equation}
So our superman's speed is very high.

\section{Problem 33}
(a)
\begin{equation}
v_{min}=\frac{c}{n}=2.26*10^8m/s
\end{equation}
(b)
\begin{equation}
K_{min}=\frac{mc^2}{\sqrt{1-\beta^2}}-mc^2
\end{equation}
For the proton:
\begin{equation}
K_{min}=484.7MeV
\end{equation}

For the pion:
\begin{equation}
K_{min}=72.0MeV
\end{equation}

(c)
\begin{equation}
E_{\pi}=\frac{m_{\pi}c^2}{\sqrt{1-\beta^2}}
\end{equation}
We can solve the $\beta=0.937$.
\begin{equation}
\theta=arccos(\frac{c}{nv})=arccos(\frac{1}{n\beta})=36.6^{\circ}
\end{equation}



\section{Problem 35}
Use formula $p(GeV)=0.3B(T)R(m)$.\\
(a)In solar system:
\begin{equation}
p=3000GeV/c
\end{equation}
We need consider relativity effect:
\begin{equation}
E=\sqrt{p^2c^2+m^2c^4}=3000GeV
\end{equation}
(b)In galaxy system:
\begin{equation}
\begin{aligned}
&p=1.5*10^{10}GeV/c\\
&E=pc=1.5*10^{10}GeV
\end{aligned}
\end{equation}

\section{Additional problem 1}
We consider a $\delta$ function:
\begin{equation}
\begin{aligned}
F(p)&=\int F(p^{'})\delta^{(3)}(p-p^{'})d^3p^{'}\\
      &=\int F(p^{'})\sqrt{m^2+p^{'2}}\delta^{(3)}(p-p^{'})\frac{d^3p^{'}}{\sqrt{m^2+p^{'2}}}
\end{aligned}
\end{equation}


Because $\frac{d^3p^{'}}{\sqrt{m^2+p^{'2}}}$ is a lorentz invariant,$\sqrt{m^2+p^{'2}}\delta^{(3)}(p-p^{'})$ is also a lorentz invariant. 

\section{Additional problem 2}
For two body decay:
\begin{equation}
\begin{aligned}
\int d\Phi_2&=\int(2\pi)^4\delta^{(4)}(p_i-p_f)\frac{d^3p_1}{(2\pi)^32E_1}\frac{d^3p_2}{(2\pi)^32E_2}\\
&=\int(2\pi)^4\delta^{(3)}(p_1+p_2)\delta(E_1+E_2-\sqrt{s})\frac{d^3p_1}{(2\pi)^32E_1}\frac{d^3p_2}{(2\pi)^32E_2}\\
&=\int\frac{1}{4(2\pi)^2}\delta(\sqrt{p^2+m_1^2}+\sqrt{p^2+m_2^2}-\sqrt{s})\frac{d^3p}{\sqrt{p^2+m_1^2}\sqrt{p^2+m_2^2}}\\
&=\int\frac{1}{4(2\pi)^2}[\frac{\delta(p-p_0)E_1E_2}{p_0(E_1+E_2)}-\frac{\delta(p+p_0)E_1E_2}{p_0(E_1+E_2)}]\frac{d^3p}{\sqrt{p^2+m_1^2}\sqrt{p^2+m_2^2}}\\
&=\int\frac{\delta(p-p_0)-\delta(p+p_0)}{16\pi^2\sqrt{s}p_0}d^3p\\
&=\frac{p_0}{4\pi\sqrt{s}}
\end{aligned}
\end{equation}
and:
\begin{equation}
p_0=\frac{(s^2+m_2^4+m_1^4-2sm_2^2-2sm_1^2-2m_1^2m_2^2)^{\frac{1}{2}}}{2\sqrt{s}}
\end{equation}

For three body decay:
\begin{equation}
\begin{aligned}
\int d\Phi_2&=\int(2\pi)^4\delta^{(4)}(p_i-p_f)\frac{d^3p_1}{(2\pi)^32E_1}\frac{d^3p_2}{(2\pi)^32E_2}\frac{d^3p_3}{(2\pi)^32E_3}\\
&=\int\frac{1}{8(2\pi)^5}\delta^{(3)}(p_1+p_2+p_3)\delta(\sqrt{s}-E_1-E_2-E_3)\frac{d^3p_1}{E_1}\frac{d^3p_2}{E_2}\frac{d^3p_3}{E_3}\\
&=\int\frac{1}{8(2\pi)^5}\delta(\sqrt{s}-\sqrt{p_1^2+m_1^2}-\sqrt{p_2^2+m_2^2}-\sqrt{p_1^2+p_2^2+2p_1p_2cos(\theta)+m_3^2})\frac{d^3p_1d^3p_2}{E_1E_2E_3}
\end{aligned}
\end{equation}




















\end{CJK*}
\end{document}
