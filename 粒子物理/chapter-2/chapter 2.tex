

\documentclass{article}
\usepackage{CJK}
\usepackage{amsmath,amssymb}
\usepackage{fancyhdr}  


\begin{document}
\begin{CJK*}{GBK}{song}

\pagestyle{fancy}  
\fancyhead{} % clear all fields  
\fancyhead[R]{Particle Physics}  
\fancyhead[L]{Chenxi Gu\\ 2017311017} 
\renewcommand{\headrulewidth}{0.4pt}  
\renewcommand{\footrulewidth}{0.4pt} 



\title {chapter 2}
\author{Chenxi Gu\\2017311017}

\date{\today}

\maketitle

\section{2.2}
(a)Use chapter 1 knowledge:
\begin{equation}
\begin{aligned}
&E_{\mu}=258.3MeV\\
&E_{\nu}=235.7MeV\\
&p_{\mu}=p_{\nu}=235.7MeV/c
\end{aligned}
\end{equation}

(b)
\begin{equation}
\begin{aligned}
\sqrt{p_k^2+m_k^2}&=\sqrt{p_{\mu}^2+m_{\mu}^2}+p_{\mu}-p_k\\
p_{\mu}&=5.011GeV/c
\end{aligned}
\end{equation}


\section{2.3}
We know second photon must be back direction.
\begin{equation}
\begin{aligned}
p_{\gamma1}+p_{\gamma2}&=\sqrt{(p_{\gamma1}-p_{\gamma2})^2+m_{\pi}^2}\\
E_{\gamma2}&=30.3MeV\\
\beta_{\pi}&=0.66
\end{aligned}
\end{equation}


\section{2.4}
\begin{equation}
\begin{aligned}
&\gamma_1=47.35\\
&\gamma_2=47348.47\\
&l_1=633.6m\\
&l_2=0.633m\\
&l^{'}_1=31.2km\\
&l^{'}_2=3.12*10^7m
\end{aligned}
\end{equation}

\section{2.5}
\begin{equation}
\begin{aligned}
&\gamma_{\pi}=\frac{E}{m}=35.8\\
&l_{rest}=0.837km\\
&l_{earth}=280m
\end{aligned}
\end{equation}

\section{2.6}
Use the $p=0.3Br$:
\begin{equation}
\begin{aligned}
&p_e=12MeV/c\\
&E_{\gamma}=24MeV
\end{aligned}
\end{equation}


\section{2.7}

\begin{itemize}
\item weak interaction:$K^+\rightarrow\pi^0\pi^+,  \mu^-\rightarrow e^-\bar{\nu_e}\nu_{\mu}$
\item strong interaction:$\rho_0\rightarrow\pi^+\pi^-, \eta^0\rightarrow \pi^+\pi^-\pi^0$
\item electromagnetic interaction:$\pi^0\rightarrow\gamma\gamma$
\end{itemize}

\section{2.8}
According to definition:
\begin{equation}
P_c=\frac{dN}{d\Omega_c}=\frac{dN}{2\pi dcos\theta_c}
\end{equation}
Relativity angle transformation formula:
\begin{equation}
cos\theta_c=\frac{-\gamma\beta+\gamma cos\theta_L}{\gamma-\gamma\beta cos\theta_L}
\end{equation}
Differentiate it:
\begin{equation}
dcos\theta_c=\frac{1-\beta^2}{(1-\beta cos\theta_c)^2}dcos\theta_L
\end{equation}
Final, we have:
\begin{equation}
P_L=\frac{dN}{2\pi dcos\theta_L}=\frac{dN}{2\pi dcos\theta_c}\frac{dcos\theta_c}{dcos\theta_L}=P_c\frac{1-\beta^2}{(1-\beta cos\theta_c)^2}
\end{equation}

\section{2.9}
The Geiger Counter's time resolution is limited($\mu s$), but the pion lifetime is $ns$.

\section{2.10}
The magnetic moment of leptons are:
\begin{equation}
\mu=\frac{gqhs}{2m}
\end{equation}
So we can get:
\begin{equation}
\begin{aligned}
&\frac{\mu_e}{\mu_{\tau}}=3477\\
&\frac{\mu_e}{\mu_{\mu}}=206
\end{aligned}
\end{equation}



\section{2.11}
\begin{equation}
\begin{aligned}
\sqrt{m^2+p_1^2}+\sqrt{m^2+p_2^2}&=\sqrt{16m^2+(p_2-p_1)^2}\\
E&=5.6GeV
\end{aligned}
\end{equation}

\section{2.13}
We know the $\pi^-$ beam in the hydrogen bubble chamber:
\begin{equation}
\pi^-+p=K^0+\Lambda^0
\end{equation}
There are two kind of neutral particle:$K^0 and \Lambda^0$,followed by the decays:
\begin{equation}
\begin{aligned}
&K^0\rightarrow\pi^++\pi^-\\
&\Lambda^0\rightarrow p+\pi^-
\end{aligned}
\end{equation}
now we know the negative one is $\pi^-$, And $p_{tot}=1998MeV,E_{\pi^-}=1905MeV$.\\
If the neutral particle is $\Lambda^0$:
\begin{equation}
\begin{aligned}
&E=2303MeV\\
&m_{\Lambda}=\sqrt{E^2-p^2}=1145MeV
\end{aligned}
\end{equation}

If the neutral particle is $K^0$:
\begin{equation}
\begin{aligned}
&E=2089MeV\\
&m_{\Lambda}=\sqrt{E^2-p^2}=612MeV
\end{aligned}
\end{equation}
So the neutral particle is $\Lambda^0$















\section{2.14}
(1) $\nu_e+n\rightarrow e^-+p$\\
\begin{equation}
\begin{aligned}
&p+m_n=\sqrt{(m_e+m_p)^2+p^2}\\
\end{aligned}
\end{equation}
No solution,this process can't be happen.\\
(2)  $\nu_{\mu}+n\rightarrow\mu^-+p$
\begin{equation}
\begin{aligned}
&p+m_n=\sqrt{(m_{\mu}+m_p)^2+p^2}\\
&E=110.2MeV
\end{aligned}
\end{equation}
(3) $\nu_{\tau}+n\rightarrow\tau^-+p$
\begin{equation}
\begin{aligned}
&p+m_n=\sqrt{(m_{\tau}+m_p)^2+p^2}\\
&E=3454.0MeV
\end{aligned}
\end{equation}






\section{2.17}
kinetic energy:
\begin{equation}
k=\sqrt{m^2+p^2}-m
\end{equation}
(a)proton:
\begin{equation}
k=0.28MeV
\end{equation}
(b)positron:
\begin{equation}
k=22.5MeV
\end{equation}














\end{CJK*}
\end{document}
