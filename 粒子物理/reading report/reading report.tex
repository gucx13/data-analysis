%%%%%%%%%%%%%%%%%%%%%%%%%%%%%%%%%%%%%%%%%
% Arsclassica Article
% LaTeX Template
% Version 1.1 (1/8/17)
%
% This template has been downloaded from:
% http://www.LaTeXTemplates.com
%
% Original author:
% Lorenzo Pantieri (http://www.lorenzopantieri.net) with extensive modifications by:
% Vel (vel@latextemplates.com)
%
% License:
% CC BY-NC-SA 3.0 (http://creativecommons.org/licenses/by-nc-sa/3.0/)
%
%%%%%%%%%%%%%%%%%%%%%%%%%%%%%%%%%%%%%%%%%

%----------------------------------------------------------------------------------------
%	PACKAGES AND OTHER DOCUMENT CONFIGURATIONS
%----------------------------------------------------------------------------------------

\documentclass[
10pt, % Main document font size
a4paper, % Paper type, use 'letterpaper' for US Letter paper
oneside, % One page layout (no page indentation)
%twoside, % Two page layout (page indentation for binding and different headers)
headinclude,footinclude, % Extra spacing for the header and footer
BCOR5mm, % Binding correction
]{scrartcl}

\input{structure.tex} % Include the structure.tex file which specified the document structure and layout

\hyphenation{Fortran hy-phen-ation} % Specify custom hyphenation points in words with dashes where you would like hyphenation to occur, or alternatively, don't put any dashes in a word to stop hyphenation altogether

%----------------------------------------------------------------------------------------
%	TITLE AND AUTHOR(S)
%----------------------------------------------------------------------------------------

\title{\normalfont\spacedallcaps{Elementary particle}} % The article title

%\subtitle{Subtitle} % Uncomment to display a subtitle

\author{\spacedlowsmallcaps{Chenxi Gu*}} % The article author(s) - author affiliations need to be specified in the AUTHOR AFFILIATIONS block

\date{} % An optional date to appear under the author(s)

%----------------------------------------------------------------------------------------

\begin{document}

%----------------------------------------------------------------------------------------
%	HEADERS
%----------------------------------------------------------------------------------------

\renewcommand{\sectionmark}[1]{\markright{\spacedlowsmallcaps{#1}}} % The header for all pages (oneside) or for even pages (twoside)
%\renewcommand{\subsectionmark}[1]{\markright{\thesubsection~#1}} % Uncomment when using the twoside option - this modifies the header on odd pages
\lehead{\mbox{\llap{\small\thepage\kern1em\color{halfgray} \vline}\color{halfgray}\hspace{0.5em}\rightmark\hfil}} % The header style

\pagestyle{scrheadings} % Enable the headers specified in this block

%----------------------------------------------------------------------------------------
%	TABLE OF CONTENTS & LISTS OF FIGURES AND TABLES
%----------------------------------------------------------------------------------------

\maketitle % Print the title/author/date block

\setcounter{tocdepth}{2} % Set the depth of the table of contents to show sections and subsections only

\tableofcontents % Print the table of contents

\listoffigures % Print the list of figures

\listoftables % Print the list of tables

%----------------------------------------------------------------------------------------
%	ABSTRACT
%----------------------------------------------------------------------------------------

\section*{Abstract} % This section will not appear in the table of contents due to the star (\section*)

In this paper we review some elementary particle.They are very important part of standard model.Such as $\pi$,$K$.Mass, Width, Angular momentum, Parity, Isospin are our point.They may interact through the strong electromagnetic weak or through some unknown force.The purpose of this review is to provide a guide for future searches what is known, what is not known. This is very necessary for the beginner.


%----------------------------------------------------------------------------------------
%	AUTHOR AFFILIATIONS
%----------------------------------------------------------------------------------------

\let\thefootnote\relax\footnotetext{* \textit{ Department of engineering physics, Tsinghua University, Pekin, China}}



%----------------------------------------------------------------------------------------

%\newpage % Start the article content on the second page, remove this if you have a longer abstract that goes onto the second page

%----------------------------------------------------------------------------------------
%	INTRODUCTION
%----------------------------------------------------------------------------------------



 
%----------------------------------------------------------------------------------------
%	PARTICLE TREE
%----------------------------------------------------------------------------------------

\section{Particle Tree}
There are so many elementary particles, So the best way is classify them.All elementary particles are made up quarks.In this paper we just focus on the meson and baryon which composed of 2 or 3 quarks.

\begin{enumerate}[noitemsep] % [noitemsep] removes whitespace between the items for a compact look
\item Light Unflavored Mesons
\item Strange Mesons
\item N Baryons
\item $\Delta$ Baryons
\item $\Lambda$ Baryons
\item $\Sigma$ Baryons
\item $\Xi$ Baryons
\end{enumerate}

%------------------------------------------------

\subsection{Light Unflavored Mesons}
What is light unflavored mesons? In the quantum mechanic, we can use some quantum numbers to describe a quantum system.For the elementary particles, we usually use $S$, $C$ and $B$. Light unflavored mesons is $S=C=B=0$.

\begin{table}[hbt]
\caption{Light Unflavored Mesons}
\centering
\begin{tabular}{lllr}
\toprule
Particle & Mass(MeV) & Width & $I^G(J^{PC})$ \\
\midrule
$\pi^{\pm}$ & $139.57018\pm0.00035$ & $(2.6033\pm0.0005)*10^{-8}s$& $1^-(0^-)$ \\
$\pi^0$ & $134.9766\pm0.0006$ &$(8.52\pm0.18)*10^{-17}s$ & $1^-(0^{-+})$ \\
$\eta$ & $547.862\pm0.017$ & $1.31\pm0.05keV$ &$0^+(0^{-+})$\\
$\eta^{'}$ & $957.78\pm0.06$ & $0.197\pm0.009MeV$ &$0^+(0^{-+})$ \\
$\rho$ &$775.26\pm0. 25$ & $149.1\pm0.8MeV$&$1^+(1^{--})$ \\
$\omega$ & $782.65\pm0.12$ & $8.49\pm0.08MeV$ &$0^-(1^{--})$ \\
$\phi$ & $1019.461\pm0.019$ & $4.266\pm0.031MeV$ &$0^-(1^{--})$ \\
\bottomrule
\end{tabular}
\label{tab:label}
\end{table}
Some particles are not the C eigenstate, such as $\pi^{\pm}$. We also could use lifetime to express the width, because we have $\Gamma=\frac{\hbar}{\tau}$.


\subsection{Strange Mesons}
Strange mesons are $C=B=0, S=\pm1$.
\begin{table}[hbt]
\caption{Strange Mesons}
\centering
\begin{tabular}{lllr}
\toprule
Particle & Mass(MeV) & Width & $I(J^{P})$ \\
\midrule
$K^{\pm}$ & $493.667\pm0.016$ & $(1.2380\pm0.0020)*10^{-8}s$ &  $\frac{1}{2}(0^-)$\\
$K^0$ & $497.611\pm0.013$ &- & $\frac{1}{2}(0^-)$ \\
$K^{*\pm}$ & $892.66\pm0.26$ & $46.2\pm1.3MeV$ &  $\frac{1}{2}(1^-)$\\
$K^{*0}$ & $895.81\pm0.19$ & $47.4\pm0.6MeV$ &  $\frac{1}{2}(1^-)$\\
\bottomrule
\end{tabular}
\label{tab:label}
\end{table}\\
Koan is not G and C eigenstate. $K^0$ is not lifetime eigenstate, but $K^0_L$ and $K^0_S$ is.

\subsection{N Baryons}
N baryons are $I=\frac{1}{2}, S=0$.
\begin{table}[hbt]
\caption{N Baryons}
\centering
\begin{tabular}{lllr}
\toprule
Particle & Mass(MeV) & Width & $I(J^{P})$ \\
\midrule
$p$ & $938.272081\pm0.000006$ & $2.1*10^{29}years$ &  $\frac{1}{2}(\frac{1}{2}^-)$\\
$n$ & $939.565413\pm0.000006$ &$880.2\pm1.0s$ & $\frac{1}{2}(\frac{1}{2}^-)$ \\
\bottomrule
\end{tabular}
\label{tab:label}
\end{table}\\
$p$ and $n$ are not G and C eigenstate. 

\subsection{$\Delta$ Baryons}
$\Delta$ baryons are $I=\frac{3}{2}, S=0$.
\begin{table}[hbt]
\caption{$\Delta$ Baryons}
\centering
\begin{tabular}{lllr}
\toprule
Particle & Mass(MeV) & Width & $I(J^{P})$ \\
\midrule
$\Delta^-$      &     -     &      -      &     $\frac{3}{2}(\frac{3}{2}^-)$\\
$\Delta^0$     &     -     &      -      &     $\frac{1}{2}(\frac{3}{2}^+)$\\
$\Delta^+$     &     -     &      -      &     $\frac{1}{2}(\frac{3}{2}^+)$\\
$\Delta^{++}$      &     -       &      -      &     $\frac{3}{2}(\frac{3}{2}^+)$\\
\bottomrule
\end{tabular}
\label{tab:label}
\end{table}\\
The pdg only give Breit-Wigner mass(mixed charges) = 1230 to 1234 MeV. And Breit-Wigner width(mixed charges) = 114 to 120 MeV.



\subsection{$\Lambda$ Baryons}
$\Lambda$ baryons are $I=0, S=-1$.
\begin{table}[hbt]
\caption{$\Lambda$ Baryons}
\centering
\begin{tabular}{lllr}
\toprule
Particle & Mass(MeV) & Width & $I^(J^{P})$ \\
\midrule
$\Lambda$      &     $1115.683\pm0.006$     &      $(2.632\pm0.020)*10^{-10}s$      &     $0(\frac{1}{2}^+)$\\
\bottomrule
\end{tabular}
\label{tab:label}
\end{table}\\



\subsection{$\Sigma$ Baryons}
$\Sigma$ baryons are $I=1, S=-1$.
\begin{table}[hbt]
\caption{$\Sigma$ Baryons}
\centering
\begin{tabular}{lllr}
\toprule
Particle & Mass(MeV) & Width & $I^(J^{P})$ \\
\midrule
$\Sigma^+$      &     $1189.37\pm0.07$     &      $(0.8018\pm0.0026)*10^{-10}s$      &     $1(\frac{1}{2}^+)$\\
$\Sigma^0$      &     $1192.642\pm0.024$     &      $(7.4\pm0.7)*10^{-20}s$      &     $1(\frac{1}{2}^+)$\\
$\Sigma^-$      &     $1197.449\pm0.030$     &      $(1.479\pm0.011)*10^{-10}s$      &     $1(\frac{1}{2}^+)$\\
$\Sigma(1385)^+$      &     $1382.80\pm0.35$     &      $36.0\pm0.7MeV$      &     $1(\frac{3}{2}^+)$\\
$\Sigma(1385)^0$      &     $1383.7\pm1.0$     &      $36\pm5MeV$      &     $1(\frac{3}{2}^+)$\\
$\Sigma(1385)^-$      &     $1387.2\pm0.5$     &      $39.4\pm2.1MeV$      &     $1(\frac{3}{2}^+)$\\
\bottomrule
\end{tabular}
\label{tab:label}
\end{table}\\



\subsection{$\Xi$ Baryons}
$\Xi$ baryons are $I=\frac{1}{2}, S=-2$.
\begin{table}[hbt]
\caption{$\Xi$ Baryons}
\centering
\begin{tabular}{lllr}
\toprule
Particle & Mass(MeV) & Width & $I(J^{P})$ \\
\midrule
$\Xi^0$      &     $1314.86\pm0.20$     &      $(2.90\pm0.09)*10^{-10}s$      &     $\frac{1}{2}(\frac{1}{2}^+)$\\
$\Xi^-$      &     $1321.71\pm0.07$     &      $(1.639\pm0.015)*10^{-10}s$      &     $\frac{1}{2}(\frac{1}{2}^+)$\\
$\Xi(1530)^0$      &     $1531.80\pm0.32$     &      $9.1\pm0.5MeV$      &     $\frac{1}{2}(\frac{3}{2}^+)$\\
$\Xi(1530)^-$      &     $1535.0\pm0.6$     &      $9.9^{+1.7}_{-1.9}MeV$      &     $\frac{1}{2}(\frac{3}{2}^+)$\\
\bottomrule
\end{tabular}
\label{tab:label}
\end{table}\\








%------------------------------------------------



\subsection{Born in Lab}

\paragraph{Pion}The invariant differential cross sections for inclusive neutral pion at mid-rapidity are measured in proton-proton collisions at $\sqrt{2} = 8 TeV$ using the ALICE detector at LHC. The neutral pion is identified from the invariant mass of photon pairs detected by the PHOS detector covering $260 < \phi < 320$, and $ |\eta| < 0.12$\cite{Yano:2310803}.
\paragraph{Kaon}A search for CP and P violation using triple-product asymmetries is performed with $\Lambda^{0}_{b}\to pK^{-}\pi^{+}\pi^{-}$,$\Lambda^{0}_{b}\to pK^{-}K^{+}K^{-}$ and $\Xi^{0}_{b}\to pK^{-}K^{-}\pi^{+}$ decays. The data sample corresponds to integrated luminosities of $1.0fb{-1}$ and $2.0fb^{-1}$, recorded with the LHCb detector at centre-of-mass energies of $7 TeV$ and $8 TeV$, respectively. The CP- and P-violating asymmetries are measured both integrating over all phase space and in specific phase-space regions. No significant deviation from CP or P symmetry is found.\cite{Aaij:2317224}
\paragraph{$\eta$ meson} We report the first observation of the doubly Cabibbo-suppressed decays $D^+\rightarrow K^+\eta^-$ using a $791fb^{-1}$ data sample collected with the Belle detector at the KEKB asymmetric-energy $e^+e^-$ collider. \cite{PhysRevLett.107.221801}
\paragraph{$\rho$ meson}\cite{Acharya:2316135}




%----------------------------------------------------------------------------------------
%	DECAY MODEL
%----------------------------------------------------------------------------------------
\section{Decay Model}

\subsection{Strong Decay}

\subsection{Weak Decay}

\subsection{Electromagnetic Decay}


%----------------------------------------------------------------------------------------
%	PARTICLE IN THE DETECTOR
%----------------------------------------------------------------------------------------
\section{Particle in the Detector}


%----------------------------------------------------------------------------------------
%	SUMMARY AND DISCUSSION
%----------------------------------------------------------------------------------------

\section{Summary and Discussion}

Reference to Figure~\vref{fig:gallery}. % The \vref command specifies the location of the reference

\begin{figure}[tb]
\centering 
\includegraphics[width=0.5\columnwidth]{GalleriaStampe} 
\caption[An example of a floating figure]{An example of a floating figure (a reproduction from the \emph{Gallery of prints}, M.~Escher,\index{Escher, M.~C.} from \url{http://www.mcescher.com/}).} % The text in the square bracket is the caption for the list of figures while the text in the curly brackets is the figure caption
\label{fig:gallery} 
\end{figure}

\lipsum[10] % Dummy text

%------------------------------------------------

\subsection{Subsection}

\lipsum[11] % Dummy text

\subsubsection{Subsubsection}

\lipsum[12] % Dummy text

\begin{description}
\item[Word] Definition
\item[Concept] Explanation
\item[Idea] Text
\end{description}

\lipsum[12] % Dummy text

\begin{itemize}[noitemsep] % [noitemsep] removes whitespace between the items for a compact look
\item First item in a list
\item Second item in a list
\item Third item in a list
\end{itemize}

\subsubsection{Table}



\begin{table}[hbt]
\caption{Table of Grades}
\centering
\begin{tabular}{llr}
\toprule
\multicolumn{2}{c}{Name} \\
\cmidrule(r){1-2}
First name & Last Name & Grade \\
\midrule
John & Doe & $7.5$ \\
Richard & Miles & $2$ \\
\bottomrule
\end{tabular}
\label{tab:label}
\end{table}

Reference to Table~\vref{tab:label}. % The \vref command specifies the location of the reference

%------------------------------------------------

\subsection{Figure Composed of Subfigures}

Reference the figure composed of multiple subfigures as Figure~\vref{fig:esempio}. Reference one of the subfigures as Figure~\vref{fig:ipsum}. % The \vref command specifies the location of the reference


%----------------------------------------------------------------------------------------
%	BIBLIOGRAPHY
%----------------------------------------------------------------------------------------

\renewcommand{\refname}{\spacedlowsmallcaps{References}} % For modifying the bibliography heading

\bibliographystyle{unsrt}

\bibliography{sample.bib} % The file containing the bibliography











%----------------------------------------------------------------------------------------

\end{document}