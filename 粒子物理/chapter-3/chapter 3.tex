

\documentclass{article}
\usepackage{CJK}
\usepackage{amsmath,amssymb}
\usepackage{fancyhdr}  


\begin{document}
\begin{CJK*}{GBK}{song}

\pagestyle{fancy}  
\fancyhead{} % clear all fields  
\fancyhead[R]{Particle Physics}  
\fancyhead[L]{Chenxi Gu\\ 2017311017} 
\renewcommand{\headrulewidth}{0.4pt}  
\renewcommand{\footrulewidth}{0.4pt} 



\title {chapter 3}
\author{Chenxi Gu\\2017311017}

\date{\today}

\maketitle

\section{3.1}
\begin{tabular}{|c|c|c|c|c|c|c|c|c|c|c|}
\hline  
     & I & $I_z$ & S & B & L & T & C & P & J & $J_3$\\
\hline  
S  & Y & Y &&&&&&&&\\
\hline %
EM  &&&&&&&&&&\\
\hline
W  &&&&&&&&&&\\
\hline
\end{tabular}

\section{3.4}

\begin{itemize}
\item $\pi^-+p\rightarrow\pi^0+n$  :  strong interaction
\item $\pi^+\rightarrow\mu^++\nu_{\mu}$  :  weak interaction
\item $\pi^+\rightarrow\mu^++\bar{\nu}_{\mu}$  :  violate $L_{\mu}$ conservation law
\item $\pi^0\rightarrow \gamma+\gamma$  :  electromagnetic interaction
\item $\pi^0\rightarrow \gamma+\gamma+\gamma$  :  violate charge conjugation.
\item $e^+e^-\rightarrow \gamma$  :  violate momentum and energy conservation law
\item $p(uud)+\bar{p}(\bar{u}\bar{u}\bar{d})\rightarrow \Lambda(uds)+\Lambda(uds)$  :  violate B and S
\item $p(uud)+p(uud)\rightarrow\Sigma^+(uus)+\pi^+(u\bar{d})$  :  violate B and S
\item $n\rightarrow p+e^-$  :  violate $L_e$ and $J$ and $J_3$ conservation law
\item $n\rightarrow p+\pi^-$  :  violate energy conservation law
\end{itemize}


\section{3.5}
\begin{itemize}
\item $\mu^+\rightarrow e^++\gamma$  :  violate $L_e$ and $L_\mu$ conservation law
\item $e^-\rightarrow \nu_e+\gamma$  :  violate charge conservation
\item $p+p\rightarrow\Sigma^++K^+$  :  violate B
\item $p+p\rightarrow p+\Sigma^++K^-$  :  violate charge conservation and S
\item $p\rightarrow e^++\nu_e$  :  violate baryon conservation
\item $p+p\rightarrow\Lambda+\Sigma^+$  :  violate charge and S conservation 
\item $p+n\rightarrow\Lambda+\Sigma^+$  :  violate S  conservation
\item $p+n\rightarrow \Xi^0(uss)+p$ :  violate S conservation
\item $p\rightarrow n+e^++\nu_e$  :  violate energy conservation
\item $n\rightarrow p+e^-+\nu_e$  :  violate $L_e$ conservation
\end{itemize}

\section{3.6}
\begin{itemize}
\item $n\rightarrow p+e^-$  :  violate $L_e$ conservation
\item $n\rightarrow \pi^++e^-$  :  violate $L_e$ conservation
\item $n\rightarrow p+\pi^-$  :  violate energy conservation
\item $n\rightarrow p+\gamma$  :  violate charge conservation
\end{itemize}


\section{3.7}
\begin{itemize}
\item $\pi^-+p\rightarrow K^-+p$  :  forbidden by S conservation
\item $\pi^-+p\rightarrow K^++\Sigma^-$  :  allowed
\item $K^-+p\rightarrow K^++\pi^-+\Xi^0$  :  allowed
\item $K^++p\rightarrow K^-+\pi^-+\Xi^0$  :  forbidden by charge and S conservation 
\end{itemize}





\section{3.8}
\begin{itemize}
\item $p\rightarrow n+e^+$  :  violate $L_e$ and energy conservation
\item $\mu^+\rightarrow\nu_{\mu}+e^+$  :  violate $L_e$ and $L_{\mu}$ conservation
\item $e^++e^-\rightarrow \nu_{\mu}+\bar{\nu}_{\mu}$  :  allowed
\item $\nu_{\mu}+p\rightarrow \mu^++n$  :  violate $L_{\mu}$ conservation 
\item $\nu_{\mu}+n\rightarrow \mu^-+p$  :  allowed
\item $\nu_{\mu}+n\rightarrow e^-+p$  :  violate $L_{\mu}$ and $L_e$ conservation 
\item $e^++n\rightarrow p+\nu_e$  :  violate $L_e$ conservation
\item $e^-+p\rightarrow n+\nu_e$  :  allowed
\end{itemize}

\section{3.9}
\begin{equation}
\begin{aligned}
&\pi^+p=\left|\frac{3}{2},+\frac{3}{2}\right\rangle\\
&\pi^-p=\sqrt{\frac{1}{3}}\left|\frac{3}{2},-\frac{1}{2}\right\rangle-\sqrt{\frac{2}{3}}\left|\frac{1}{2},-\frac{1}{2}\right\rangle\\
&K^0\Sigma^0=\sqrt{\frac{2}{3}}\left|\frac{3}{2},-\frac{1}{2}\right\rangle+\sqrt{\frac{1}{3}}\left|\frac{1}{2},-\frac{1}{2}\right\rangle\\
&K^+\Sigma^-=\sqrt{\frac{1}{3}}\left|\frac{3}{2},-\frac{1}{2}\right\rangle-\sqrt{\frac{2}{3}}\left|\frac{1}{2},-\frac{1}{2}\right\rangle\\
&K^+\Sigma^+=\left|\frac{3}{2},+\frac{3}{2}\right\rangle
\end{aligned}
\end{equation}
With a proportionality constant N equal for all we obtain
\begin{equation}
\begin{aligned}
&\sigma(\pi^-p\rightarrow K^0\Sigma^0)=N\left|\frac{\sqrt{2}}{3}A_{3/2}-\frac{\sqrt{2}}{3}A_{1/2}\right|^2\\
&\sigma(\pi^-p\rightarrow K^+\Sigma^-)=N\left|\frac{1}{3}A_{3/2}+\frac{2}{3}A_{1/2}\right|^2\\
&\sigma(\pi^+p\rightarrow K^+\Sigma^+)=N\left|A_{3/2}\right|^2
\end{aligned}
\end{equation}
they proceed only through the $I=\frac{3}{2}$ channel:
\begin{equation}
\sigma(\pi^-p\rightarrow K^0\Sigma^0):\sigma(\pi^-p\rightarrow K^+\Sigma^-):\sigma(\pi^+p\rightarrow K^+\Sigma^+)=2:1:9
\end{equation}

\section{3.10}
Using the result in 3.9.
\begin{equation}
\begin{aligned}
&\sigma(\pi^-p\rightarrow K^0\Sigma^0):\sigma(\pi^-p\rightarrow K^+\Sigma^-):\sigma(\pi^+p\rightarrow K^+\Sigma^+)\\
=&\left|\frac{\sqrt{2}}{3}A_{3/2}-\frac{\sqrt{2}}{3}A_{1/2}\right|^2:\left|\frac{1}{3}A_{3/2}+\frac{2}{3}A_{1/2}\right|^2:\left|A_{3/2}\right|^2
\end{aligned}
\end{equation}

\section{3.11}

\begin{equation}
\begin{aligned}
&\pi^-p=\sqrt{\frac{1}{3}}\left|\frac{3}{2},-\frac{1}{2}\right\rangle-\sqrt{\frac{2}{3}}\left|\frac{1}{2},-\frac{1}{2}\right\rangle\\
&\pi^+n=\sqrt{\frac{1}{3}}\left|\frac{3}{2},+\frac{1}{2}\right\rangle-\sqrt{\frac{2}{3}}\left|\frac{1}{2},+\frac{1}{2}\right\rangle\\
&\Lambda K^0=\left|\frac{1}{2},-\frac{1}{2}\right\rangle\\
&\Lambda K^+=\left|\frac{1}{2},+\frac{1}{2}\right\rangle\\
\end{aligned}
\end{equation}
With a proportionality constant N equal for all we obtain
\begin{equation}
\begin{aligned}
&\sigma(\pi^-p\rightarrow\Lambda K^0)=N\left|\sqrt{\frac{2}{3}}A_{1/2}\right|^2\\
&\sigma(\pi^+n\rightarrow\Lambda K^+)=N\left|\sqrt{\frac{2}{3}}A_{1/2}\right|^2\\
\end{aligned}
\end{equation}
So the ratio of cross-sections is $1:1$

\section{3.12}
Same analysis as 3.9
\begin{equation}
\sigma(p+d\rightarrow {}^3\text{He}+\pi^0):\sigma(p+d\rightarrow {}^3\text{H}+\pi^+)=1:2
\end{equation}

\section{3.13}
Same analysis as 3.9
\begin{equation}
\frac{\sigma(pp\rightarrow d\pi^+)}{\sigma(pn\rightarrow d\pi^0)}=2
\end{equation}

\section{3.14}
\begin{equation}
\frac{\sigma(K^-+{}^4\text{He}\rightarrow\Sigma^0+{}^3\text{H})}{\sigma(K^-+{}^4\text{He}\rightarrow\Sigma^-+{}^3\text{He})}=2
\end{equation}


\section{3.15}
\begin{equation}
\begin{aligned}
K^-p=\sqrt{\frac{1}{2}}\left|1,0\right\rangle-\sqrt{\frac{1}{2}}\left|0,0\right\rangle
\end{aligned}
\end{equation}



\begin{equation}
\pi^+\Sigma^-=\sqrt{\frac{1}{6}}\left|2,0\right\rangle+\sqrt{\frac{1}{2}}\left|1,0\right\rangle+\sqrt{\frac{1}{3}}\left|0,0\right\rangle
\end{equation}

\begin{equation}
\pi^0\Sigma^0=\sqrt{\frac{2}{3}}\left|2,0\right\rangle-\sqrt{\frac{1}{3}}\left|0,0\right\rangle
\end{equation}

\begin{equation}
\pi^-\Sigma^+=\sqrt{\frac{1}{6}}\left|2,0\right\rangle-\sqrt{\frac{1}{2}}\left|1,0\right\rangle+\sqrt{\frac{1}{3}}\left|0,0\right\rangle
\end{equation}

\begin{equation}
\begin{aligned}
&\sigma(K^-p\rightarrow\pi^+\Sigma^-):\sigma(K^-p\rightarrow\pi^0\Sigma^0):\sigma(K^-p\rightarrow\pi^-\Sigma^+)\\
&=\left|\frac{1}{2}A_{1}-\sqrt{\frac{1}{6}}A_{0}\right|^2:\left|\frac{1}{6}A_{0}\right|^2:\left|\frac{1}{2}A_{1}+\sqrt{\frac{1}{6}}A_{0}\right|^2
\end{aligned}
\end{equation}

\section{3.16}
Using the result of 3.11:
\begin{equation}
\sigma(\pi^-p\rightarrow\pi^-p)=N\left|\frac{1}{3}A_{3/2}+\frac{2}{3}A_{1/2}\right|^2
\end{equation}
And the spin for $\pi^0n$ system.
\begin{equation}
\begin{aligned}
&\pi^0n=\sqrt{\frac{2}{3}}\left|\frac{3}{2},-\frac{1}{2}\right\rangle+\sqrt{\frac{1}{3}}\left|\frac{1}{2},-\frac{1}{2}\right\rangle\\\
&\sigma(\pi^-p\rightarrow\pi^0n)=N\left|\frac{\sqrt{2}}{3}A_{3/2}-\frac{\sqrt{2}}{3}A_{1/2}\right|^2
\end{aligned}
\end{equation}
So
\begin{equation}
\frac{\sigma(\pi^-p\rightarrow\pi^-p)}{\sigma(\pi^-p\rightarrow\pi^0n)}=\frac{\left|\frac{1}{3}A_{3/2}+\frac{2}{3}A_{1/2}\right|^2}{\left|\frac{\sqrt{2}}{3}A_{3/2}-\frac{\sqrt{2}}{3}A_{1/2}\right|^2}
\end{equation}


\section{3.17}
(a)For the S wave : $P_{\pi^-d}=-1$, the $nn$ system orbital momentum is 1. And the $nn$ system wave function must be antisymmetric. Since the spatial part is antisymmetric, the  spin function is symmetric. We can get S=1\\
(b)For the P wave : $P_{\pi^-d}=+1$, the $nn$ system orbital momentum is 0 or 2. The total spin is 0.



\section{3.18}
(1)
\begin{equation}
C=(-1)^{l+s}
\end{equation}
(2)
\begin{equation}
(-1)^{l+s}=(-1)^n
\end{equation}
(3)For the ortho-positronium minimum number photon is 3,para-positronium minimum number photons are 2.

\section{3.19}
(1)
\begin{equation}
C(\bar{p}p)=(-1)^{l+s}=C(n\pi^0)=+1
\end{equation}
So the state are ${}^0S_1,{}^3P_0,{}^3P_1,{}^3P_2,{}^1D_2$\\\\
(2)
$P(2\pi^0)$ must be symmetric the $L$ is oven, for the $2\pi^0$ system  $J=L$ is oven.We also know $P(\bar{p}p)=(-1)^{L+1}$, L is odd.  ${}^3P_2$ and ${}^3P_0$ satisfy the condition.

\section{3.20}
Because the $I=0$ is symmetric, $P(\pi^+\pi^-)=(-1)^l$ l is even.\\
For $I=1$ is antisymmetric, l is odd.



\section{3.21}
We know $\bar{p}p$ system $P$ and $C$:
\begin{equation}
\begin{aligned}
&P(\bar{p}p)=(-1)^{l+1}\\
&C(\bar{p}p)=(-1)^{l+s}
\end{aligned}
\end{equation}

(a)S wave we can get:
\begin{equation}
\begin{aligned}
&P(\pi^+\pi^-)=+1\\
&C(\pi^+\pi^-)=+1\\
&J_{\pi^+\pi^-}=0
\end{aligned}
\end{equation}
So only ${}^3P_0$ is allowed.
(b)Follow the analysis in (a).${}^3S_1, {}^3D_1$ are allowed.
(c)${}^3P_2$ is allowed.


\section{3.22}
\begin{itemize}
\item $\pi^+p\rightarrow D^+p$ : is allowed.
\item $\pi^+p\rightarrow D^-\Lambda_c\pi^+\pi^+$ : is allowed.
\item $\pi^+p\rightarrow D^+\Lambda_c$ : is allowed.
\item $\pi^+p\rightarrow D^-\Lambda_c$ : is not allowed.
\end{itemize}

\section{3.23}
\begin{itemize}
\item $\pi^-p\rightarrow D^0\Lambda_b$ : is allowed.
\item $\pi^-p\rightarrow B^0\Lambda_b$ : is allowed.
\item $\pi^-p\rightarrow B^+\Lambda_b\pi^-$ : is allowed.
\item $\pi^-p\rightarrow B^-\Lambda_b\pi^+$ : is allowed.
\item $\pi^-p\rightarrow B^-B^+$ : is not allowed.
\end{itemize}

\section{3.25}
(a)If $I=\frac{3}{2}$
\begin{equation}
\frac{\sigma(\Delta^0\rightarrow p\pi^-)}{\sigma(\Delta^0\rightarrow n\pi^0)}=1:2
\end{equation}

(b)If $I=\frac{1}{2}$
\begin{equation}
\frac{\sigma(\Delta^0\rightarrow p\pi^-)}{\sigma(\Delta^0\rightarrow n\pi^0)}=2:1
\end{equation}







\section{3.27}
\begin{itemize}
\item $\mu^-\rightarrow e^-+\gamma$  :  forbidden by $L_e$ and $L_{\mu}$ conservation.
\item $\pi^+\rightarrow \mu^++\nu_{\mu}+\bar{\nu}_{\mu}$  :  forbidden by $L_{\mu}$ conservation.
\item $\Sigma^0\rightarrow \Lambda+\gamma$  :  allowed.
\item $\eta\rightarrow \gamma+\gamma+\gamma$  :  forbidden by C conservation.
\item $\gamma+p\rightarrow\pi^0+p$  :  is allowed.
\item $p\rightarrow\pi^0+e^+$  :  forbidden by $L_e$ conservation.
\item $\pi^-\rightarrow\mu^-+\gamma$  :  forbidden by $L_{\mu}$ conservation.
\end{itemize}


\section{3.28}
\begin{itemize}
\item $\pi^-+p\rightarrow \Sigma^0+K^0$
\item $e^++n\rightarrow p+\bar{\nu}_e$
\item $\Xi^0\rightarrow \Lambda+\bar{K}^0$
\end{itemize}

\section{3.29}
We consider the $I_z$:
\begin{equation}
1+\frac{1}{2}=\frac{1}{2}+\frac{1}{2}+I_{z_{\Xi^0}}
\end{equation}
So $I_{z_{\Xi^0}}=\frac{1}{2}$.\\\\
Second,$\pi^+p=\left|\frac{3}{2},\frac{3}{2}\right\rangle$ and $K^+K^+=\left|1,1\right\rangle$.\\
The $I_{\Xi^0}$ could be $\frac{1}{2}, \frac{3}{2}, \frac{5}{2}$.



\end{CJK*}
\end{document}
