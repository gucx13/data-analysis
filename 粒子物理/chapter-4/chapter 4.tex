

\documentclass{article}
\usepackage{CJK}
\usepackage{amsmath,amssymb}
\usepackage{fancyhdr}  


\begin{document}
\begin{CJK*}{GBK}{song}

\pagestyle{fancy}  
\fancyhead{} % clear all fields  
\fancyhead[R]{Particle Physics}  
\fancyhead[L]{Chenxi Gu\\ 2017311017} 
\renewcommand{\headrulewidth}{0.4pt}  
\renewcommand{\footrulewidth}{0.4pt} 



\title {chapter 4}
\author{Chenxi Gu\\2017311017}

\date{\today}

\maketitle

\section{4.1}
\begin{equation}
G\left|\pi^0\right\rangle=-\left|\pi^0\right\rangle
\end{equation}
So $\pi^0$ system is G-parity eigenstate, and eigenvalue is -1.

\begin{equation}
G\left|\pi^+\pi^+\pi^-\right\rangle=-\left|\pi^+\pi^+\pi^-\right\rangle
\end{equation}
So $\pi^+\pi^+\pi^-$ system is G-parity eigenstate, and eigenvalue is -1.

\begin{equation}
G\left|\rho^+\right\rangle=\left|\rho^+\right\rangle
\end{equation}
So $\rho^+$ system is G-parity eigenstate, and eigenvalue is 1.


\section{4.2}
K is not the G-parity eigenstate.\\
$\phi$ is the G-parity eigenstate, and eigenvalue is -1.\\
$\eta$ is the G-parity eigenstate, and eigenvalue is 1.\\
$\omega$ is not the G-parity eignstate.
\section{4.3}
\begin{equation}
\begin{aligned}
&\left|\pi^+\pi^-\right\rangle=\sqrt{\frac{1}{6}}\left|2,0\right\rangle+\sqrt{\frac{1}{2}}\left|1,0\right\rangle+\sqrt{\frac{1}{3}}\left|0,0\right\rangle\\
&\left|\pi^0\pi^0\right\rangle=\sqrt{\frac{2}{3}}\left|2,0\right\rangle-\sqrt{\frac{1}{3}}\left|0,0\right\rangle
\end{aligned}
\end{equation}
So we know $\rho^0=\left|1,0\right\rangle$.The isospin wave function is antisymmetric. $l=J$ is odd.$P=C=-1$.
Because the decay produce two $\pi$, $G=1$.

\section{4.5}
(a)$S=-1,\quad Y=0,\quad I=1,\quad I_3=1$\\
(b)$P=P_{\pi^+}P_{\lambda^0}(-1)^L=+,\quad J=\frac{1}{2} \quad or\quad \frac{3}{2}$

\section{4.7}
$\rho\rightarrow\pi^0\pi^0$ is strong interaction.\\
forbidding reason:
\begin{itemize}
\item C violation:$C_{\pi^0\pi^0}=+1$, but $C_{\rho^0}=-1$
\item P violation:$P_{\rho^0}=-1$, but $\pi^0\pi^0$ system wave function is symmetric
\item I violation  
\end{itemize}

\section{4.8}
\begin{itemize}
\item $\rho^0\rightarrow\pi^0\gamma$ : allowed
\item $f^0\rightarrow\pi^0\gamma$ : C violation
\end{itemize}

\section{4.10}
\begin{equation}
\Gamma(K^-p)/\Gamma(\bar{K}^0n)=1
\end{equation}

\begin{equation}
\Gamma(\pi^-\pi^+)/\Gamma(\bar{K}^0n)=1
\end{equation}
\section{4.11}
\begin{tabular}{|c|c|c|c|c|c|c|}% 通过添加 | 来表示是否需要绘制竖线
\hline  % 在表格最上方绘制横线
    &   $ \bar{p}p^3S_1 $ & $ \bar{p}p^3S_1 $   & $ \bar{p}p^1S_0 $   & $ \bar{p}p^1S_0 $    & $ \bar{p}n^3S_1 $   &$ \bar{p}n^1S_0 $ \\
\hline  %在第一行和第二行之间绘制横线
$J^P$    &  $1^-$  & $1^-$   & $0^-$   & $0^-$   & $1^-$  &  $1^-$\\
\hline % 在表格最下方绘制横线
C   &  -  &  -  &  +  & +   &  X &  X\\
\hline    
I    &   0  &  1  &  0  &  1  & 1  &1\\
\hline 
G    & -   &  +  & +   &  -  &  + &  -\\
\hline 
\end{tabular}\\
\\
$G_{\pi^-\pi^-\pi^+}=-1$, so only left $^1\text{S}_0$
\begin{equation}
\sigma(\bar{p}n\rightarrow\rho^0\pi^-):\sigma(\bar{p}n\rightarrow\rho^-\pi^0)=1:1
\end{equation}
\begin{equation}
\sigma(\bar{p}p(I=1)\rightarrow\rho^+\pi^-):\sigma(\bar{p}p(I=1)\rightarrow\rho^0\pi^0):\sigma(\bar{p}p(I=1)\rightarrow\rho^-\pi^+)=1:0:1
\end{equation}

\begin{equation}
\sigma(\bar{p}p(I=0)\rightarrow\rho^+\pi^-):\sigma(\bar{p}p(I=0)\rightarrow\rho^0\pi^0):\sigma(\bar{p}p(I=0)\rightarrow\rho^-\pi^+)=1:1:1
\end{equation}



\section{4.12}
The isospin of $\pi^0\pi^0$ system might be $\left|0,0\right\rangle,\quad \left|2,0\right\rangle$ 

\section{4.13}
We can find the density 0 area in fig 4.12 on text book.
\section{4.14}
There are two kind of deuteron state:
\begin{equation}
^3S_1 \quad ^3D_1
\end{equation}
\section{4.15}
For $Q=0$:
\begin{equation}
\bar{c}\bar{d}\bar{d}(C=-1)\quad udd(C=0)\quad cdd(C=1) 
\end{equation}
For $Q=1$:
\begin{equation}
uud(C=0)\quad ucd(C=1)\quad ccd(C=2) 
\end{equation}

\section{4.16}
The quark content is $udc$.
\section{4.17}
\begin{equation}
sss\quad uuc\quad ucs\quad css\quad udb
\end{equation}

\section{4.18}
\begin{equation}
c\bar{d}\quad u\bar{c}\quad u\bar{b}\quad c\bar{b}
\end{equation}

\section{4.19}
\begin{itemize}
\item positive strangeness and negative charm : $\bar{c}\bar{s}$ is fraction charge.
\item spin 0 baryon : baryon spin is fraction.Because of the quark spin $\frac{1}{2}$
\item antibaryon with charge +2 : $\bar{q}\bar{q}\bar{q}$ max charge is +1.
\item positive meson with strangeness -1 : $Q(\bar{q}s)<>1$ no quark with charge $\frac{4}{3}$ 
\end{itemize}

\section{4.20}
Using the formula $Q=I_z+\frac{Y}{2}$, we can get charge.
\section{4.21}
\begin{itemize}
\item meson : +1,0,-1
\item baryon : +2,+1,0,-1
\end{itemize}
\section{4.22}
\begin{equation}
\tau_{J/\psi}=\frac{\hbar}{\Gamma_{J/\psi}}=7.25*10^{-21}s
\end{equation}
\begin{equation}
l=\frac{\beta ct}{\sqrt{1-\beta^2}}=3.5*10^{-12}
\end{equation}
(a)$p_J=5GeV$
\begin{equation}
\begin{aligned}
&E=2.94GeV\\
& \theta=0.55
\end{aligned}
\end{equation}


(b)$p_J=50GeV$
\begin{equation}
\begin{aligned}
&E=25.048GeV\\
& \theta=0.062
\end{aligned}
\end{equation}

\section{4.23}
We could calculate the distance between primary vertex and second vertex is 1.28mm, so we should use the silicon micro-strip detector.
\section{4.24}
\begin{equation}
\int\sigma(E)dE=\frac{6\pi^2\Gamma_e\Gamma_f}{\Gamma M_R^2}
\end{equation}

\section{4.26}
(1)The isospin of baryon is $\left|1,0\right\rangle$.\\
(2)The ratio between two observed channels is $1:1$.

\section{4.27}
\begin{itemize}
\item $-\frac{1}{2}A_{1,0}-\frac{1}{\sqrt{6}}A_{0,0}$
\item $\frac{1}{\sqrt{6}}A_{0,0}$
\item $\frac{1}{2}A_{1,0}-\frac{1}{\sqrt{6}}A_{0,0}$
\item $-\frac{1}{\sqrt{2}}A_{1,1}$
\item $\frac{1}{\sqrt{2}}A_{1,1}$
\end{itemize}

\section{4.28}
\begin{itemize}
\item forbidden by S conservation
\item allowed
\item forbidden by S conservation and charge conservation
\item forbidden by energy conservation
\item forbidden by S conservation
\item forbidden by S conservation
\item forbidden by S conservation
\end{itemize}




\end{CJK*}
\end{document}
