\documentclass{article}
\usepackage{CJK}
\usepackage{amsmath,amssymb}
\usepackage{fancyhdr}  


\begin{document}
\begin{CJK*}{GBK}{song}

\pagestyle{fancy}  
\fancyhead{} % clear all fields  
\fancyhead[R]{Group Theory}  
\fancyhead[L]{Chenxi Gu\\ 2017311017} 
\renewcommand{\headrulewidth}{0.4pt}  
\renewcommand{\footrulewidth}{0.4pt} 



\title {homework 2}
\author{Chenxi Gu\\2017311017}

\date{\today}

\maketitle

\section{2.2}
$A(g_a)$ is a representation.
\begin{equation}
A(g_1)A(g_2)=A(g_1g_2)
\end{equation}

So we can get :
\begin{equation}
A^T(g_1)^{-1}A^T(g_2)^{-1}=[[A(g_1)A(g_2)]^T]^{-1}=A^T(g_1g_2)^{-1}
\end{equation}





\begin{equation}
A^{\dagger}(g_1)^{-1}A^{\dagger}(g_2)^{-1}=[[A(g_1)A(g_2)]^{\dagger}]^{-1}=A^{\dagger}(g_1g_2)^{-1}
\end{equation}

$A^T(g_a)^{-1},A^{\dagger}(g_a)^{-1}$ are also representation.\\


$A^T(g_a),A^{\dagger}(g_a)$ are not representation,but Abel group.
if $A(g_a)$ is a Unitary representation:
\begin{equation}
\begin{aligned}
&A^{\dagger}(g)=A(g^{-1})\\
&A^{T\dagger}(g)^{-1}=A^T(g)\\
&A^{\dagger \dagger}(g)^{-1}=A(g)^{-1}
\end{aligned}
\end{equation}
So $A^T(g)^{-1},A^{\dagger}(g)^{-1}$ are unitary representation.It is esay to prove they are irreducible representation.

\section{2.3}
It is easy to find the matrix commute with all elements in group $A(g)$:
\begin{equation}
[A(g_a),\sum_CA(g_b)]=0
\end{equation} 
Using shur lemma :
\begin{equation}
\sum_CA(g_b)=\lambda E
\end{equation}

\section{2.7}
We choose natural basic :
\begin{equation}
e=
\begin{pmatrix}
     1 \\
     0 \\
     0 \\
     0 \\
\end{pmatrix}\quad
a=
\begin{pmatrix}
     0 \\
     1 \\
     0 \\
     0 \\
\end{pmatrix}\quad
a^2=
\begin{pmatrix}
     0 \\
     0 \\
     1 \\
     0 \\
\end{pmatrix}\quad
a^3=
\begin{pmatrix}
     0 \\
     0 \\
     0 \\
     1 \\
\end{pmatrix}
\end{equation}
So we can get the left representation:
\begin{equation}
L(e)=
\begin{pmatrix}
     1  &  0  &  0  &  0\\
     0  &  1  &  0  &  0\\
     0  &  0  &  1  &  0\\
     0  &  0  &  0  &  1\\
\end{pmatrix}\quad
L(a)=
\begin{pmatrix}
     0  &  0  &  0  &  1\\
     1  &  0  &  0  &  0\\
     0  &  1  &  0  &  0\\
     0  &  0  &  1  &  0\\
\end{pmatrix}
\end{equation}
The right presentation:
\begin{equation}
R(e)=
\begin{pmatrix}
     1  &  0  &  0  &  0\\
     0  &  1  &  0  &  0\\
     0  &  0  &  1  &  0\\
     0  &  0  &  0  &  1\\
\end{pmatrix}\quad
R(a)=
\begin{pmatrix}
     0  &  1  &  0  &  0\\
     0  &  0  &  1  &  0\\
     0  &  0  &  0  &  1\\
     1  &  0  &  0  &  0\\
\end{pmatrix}
\end{equation}





\section{2.8}


\section{2}
In the active view:
\begin{equation}
SO(2)=
\begin{pmatrix}
     cos(\theta)  &  -sin(\theta)  &  0  \\
      sin(\theta)  &  cos(\theta)  &  0  \\
                    0  &                 0  &  1  \\
\end{pmatrix}
\end{equation}
In the passive view:
\begin{equation}
SO(2)=
\begin{pmatrix}
     cos(\theta)  &   sin(\theta)  &  0  \\
     -sin(\theta)  &  cos(\theta)  &  0  \\
                    0  &                 0  &  1  \\
\end{pmatrix}
\end{equation}


\section{3}
We can get the $D_{2n}=\{e,a,a^2...,a^{n-1},b,ba,ba^2.....,ba^{n-1}\}$\\
Secondly, we can find the class in the group:\\
(a)$n=2k$
\begin{equation}
\begin{aligned}
\{e\}&\\
\{a,a^{n-1}\}&\\
\{a^2,a^{n-2}\}&\\
....\\
\{a_k\}&\\
\{b,b^2....,b^{2k}\}&\\
\{ba,ba^{3}...b^{2k-1}\}&\\
\end{aligned}
\end{equation}

(b)$n=2k+1$

\begin{equation}
\begin{aligned}
\{e\}&\\
\{a,a^{n-1}\}&\\
\{a^2,a^{n-2}\}&\\
....\\
\{a_k,a_{k+1}\}&\\
\{b,b^2....,b^{n}\}&\\
\end{aligned}
\end{equation}























\end{CJK*}
\end{document}

